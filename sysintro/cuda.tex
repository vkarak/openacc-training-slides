%%%%%%%%%%%%%%%%%%%%%%%%%%%%%%%%%%%%
\cscschapter{Running the code}
%%%%%%%%%%%%%%%%%%%%%%%%%%%%%%%%%%%%

%  ####   #    #  #####     ##
% #    #  #    #  #    #   #  #
% #       #    #  #    #  #    #
% #       #    #  #    #  ######
% #    #  #    #  #    #  #    #
%  ####    ####   #####   #    #
\begin{frame}[fragile]{PrgEnv-Cray: CUDA (1)}

%.........
\begin{info}{Setup}
\begin{itemize}
 \item module load      \lst{craype-accel-nvidia35} \lst{libsci\* for CCE only \!}
 \item \$CRAY\_ACCEL\_TARGET = \lst{nvidia35}
 \item \$CUDATOOLKIT\_HOME = \lst{ Not set } / export ... 
 \item craype-accel-nvidia35 = \lst{ Not set } / add -arch=sm\_37 
\end{itemize}
\end{info}
%.........

%.........
\begin{info}{Compile}
\begin{itemize}
 \item cc -c mpic.c ; nvcc \lst{-arch=sm\_37} -c mpicu.cu
 \item \lst{CC} mpic.o mpicu.o -o CRAY.MGMT 
 \item undefined reference to \lst{'\_\_gxx\_personality\_v0'} with cc (use \lst{CC})
\end{itemize}
\end{info}
%.........

\end{frame}


\begin{frame}[fragile]{Running the code: the compute node}
%.........
\begin{info}{Run}
\begin{itemize}
 \item LD\_LIBRARY\_PATH=/opt/local/slurm/default/lib64:\$...
\end{itemize}
\end{info}
%.........

%.........
\begin{code}{srun -n1 CRAY.MGMT}
\begin{lstlisting}[style=boxcudatiny]
=== get_gpu_info ===
Process 0 on prod-0001 out of 1 Device 0 (Tesla K80)

=== /proc/driver/nvidia/version ===
NVRM version: NVIDIA UNIX x86_64 Kernel Module  340.87

=== cudaGetDeviceProperties ===
Device 0: "Tesla K80"
  CUDA Driver Version / Runtime Version     6.5 / 6.5
  CUDA Capability Major/Minor version number:    3.7

srun -n1 cat /proc/cpuinfo      (HASWELL)
        processor  : 0-23
        model name : Intel(R) Xeon(R) CPU E5-2690 v3 @ 2.60GHz 

srun -n1 cat /proc/meminfo      (256G)
        MemTotal:       264572408 kB
\end{lstlisting}
\end{code}
%.........

%.........
\begin{code}{PizDaint}
\begin{lstlisting}[style=boxcudatiny]
Runtime Version = 6.5 / Capability = 3.5 / Driver = 340.87
\end{lstlisting}
\end{code}
%.........

\end{frame}




