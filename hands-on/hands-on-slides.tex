\documentclass[aspectratio=169,12pt]{beamer}
\usetheme{CSCS}

\usepackage[no-math]{fontspec}
\usepackage{xspace}

\setsansfont{Arial}
\setmonofont{Courier New}
\urlstyle{sf}

\graphicspath{{./figs/}}


% define footer text
\newcommand{\footlinetext}{Directive Based GPU Programming: OpenACC and OpenMP}

% Select the image for the title page
\newcommand{\picturetitle}{cscs_images/image3.pdf}
%\newcommand{\picturetitle}{cscs_images/image5.pdf}
%\newcommand{\picturetitle}{cscs_images/image6.pdf}

\lstdefinestyle{cxxstyle}{
  basicstyle=\scriptsize\ttfamily,
  keywordstyle=\color{blue},
  stringstyle=\color{magenta},
  language=[11]C++,
  showstringspaces=false,
  escapechar=\%
}

\lstdefinestyle{shstyle}{
  basicstyle=\scriptsize\ttfamily,
  keywordstyle=\color{blue},
  stringstyle=\color{magenta},
  commentstyle=\itshape\color{cscsred},
  language=bash,
}

\newcommand\cxxinline[2][]{\lstinline[style=cxxstyle,basicstyle=\ttfamily,#1]!#2!}
\newcommand\shinline[2][]{\lstinline[style=shstyle,basicstyle=\ttfamily,#1]!#2!}

% Please use the predifined colors:
% cscsred, cscsgrey, cscsgreen, cscsblue, cscsbrown, cscspurple, cscsyellow, cscsblack, cscswhite

\author{\emph{Vasileios Karakasis, CSCS}}
\title{Introduction to the GPU architecture}
\subtitle{Directive Based GPU Programming: OpenACC and OpenMP}
\date{29--31 May 2017}

\begin{document}

% TITLE SLIDE
\cscstitle

\begin{frame}[fragile]{Hands-on exercises}{General info}
  \begin{itemize}
  \item Available at \url{https://github.com/vkarak/openacc-training}
    \begin{itemize}
    \item Clone the repository: \shinline{git clone https://github.com/vkarak/openacc-training.git}
    \end{itemize}
  \item \shinline{exercises/} $\rightarrow$ hands-on exercises
  \item \shinline{solutions/} $\rightarrow$ solutions to exercises (will appear after the course)
  \item \shinline{slides/} $\rightarrow$ slides of the course
  \end{itemize}
\end{frame}

\begin{frame}[fragile]{Hands-on exercises}{General info}
  \begin{enumerate}
  \item \shinline{exercises/cray/Himeno}
    \begin{itemize}
    \item Himeno benchmark
    \item Basic directive concepts for accelerators
    \end{itemize}
  \item \shinline{exercises/openacc/diffusion}
    \begin{itemize}
    \item 2D diffusion example
    \item Interoperability with MPI and CUDA
    \end{itemize}
  \item \shinline{exercises/openacc/gemm}
    \begin{itemize}
    \item GEMM
    \item Interoperability with CUBLAS
    \end{itemize}
  \end{enumerate}
  \vfill
  Grep for \texttt{TODO} for places to modify
\end{frame}

\begin{frame}[fragile]{Hands-on exercises}{How to compile}
  \begin{itemize}
  \item Cray compiler
    \begin{itemize}
    \item Make sure that \shinline{PrgEnv-cray} module is loaded
    \item \shinline{module load craype-accel-nvidia60}
    \item \shinline{make CRAY=1} (default)
    \end{itemize}
  \item PGI compiler
    \begin{itemize}
    \item Make sure that \shinline{PrgEnv-pgi} module is loaded
    \item \shinline{module load cudatoolkit}
    \item \shinline{make PGI=1}
    \end{itemize}
  \end{itemize}
\end{frame}

\begin{frame}[fragile]{Hands-on exercises}{How to run}
  \begin{enumerate}
  \item Source some necessary Slurm presets
    \begin{itemize}
    \item \texttt{source <prefix>/tools/course.sh}
    \end{itemize}
  \item Launch directly the job from the command line
    \begin{itemize}
    \item \texttt{srun -Cgpu -N<nnodes> -n<ntasks> <exec> <args> ...}
    \item \texttt{OMP\_NUM\_THREADS=<nthreads> srun -Cgpu ...}
    \end{itemize}
  \item \dots or use the job submission script provided
    \begin{itemize}
    \item \shinline{sbatch job.batch}
    \item The output is written in \shinline{job.out}
    \end{itemize}
  \end{enumerate}
\end{frame}

\begin{frame}[fragile]{2D diffusion example}{Source code}
  \begin{itemize}
  \item \shinline{diffusion2d\_omp.\{cpp,f90\}}
    \begin{itemize}
    \item Single node OpenMP version for the CPU
    \end{itemize}
  \item \shinline{diffusion2d\_openacc.\{cpp,f90\}}
    \begin{itemize}
    \item Single node OpenACC version
    \end{itemize}
  \item \shinline{diffusion2d\_openacc\_mpi.\{cpp,f90\}}
    \begin{itemize}
    \item MPI+OpenACC version
    \item If \shinline{OPENACC\_DATA} is undefined, data management is performed by CUDA
    \end{itemize}
  \end{itemize}
\end{frame}


\begin{frame}[fragile]{2D diffusion example}{Executables}
  Five versions in both C++/Fortran
  \vspace\baselineskip
  \begin{itemize}
  \item \shinline{diffusion2d.omp}: OpenMP version for the CPU (baseline)
  \item \shinline{diffusion2d.openacc}: Single-node OpenACC version
  \item \shinline{diffusion2d.openacc.cuda}: Single-node OpenACC version with memory managed by CUDA
  \item \shinline{diffusion2d.openacc.mpi}: MPI+OpenACC version
  \item \shinline{diffusion2d.openacc.cuda.mpi}: MPI+OpenACC version with memory managed by CUDA
  \end{itemize}
\end{frame}

\begin{frame}[fragile]{2D diffusion example}{Goal of the exercise}
  \begin{itemize}
  \item Familiarize with data management in OpenACC
  \item Learn how to interact with CUDA pointers from OpenACC
  \item Learn how to enable the fast RDMA data path with Cray MPI
  \end{itemize}
  \vfill
  \emph{Plot the result with \shinline{plot.sh} and fill in the performance table in \shinline{performance.text}}
\end{frame}

\begin{frame}[fragile]{GEMM}
  \begin{itemize}
  \item C++ only (apologies to Fortran guys!)
  \item Versions
    \begin{itemize}
    \item Naive: directly from the math textbook (\texttt{make CPPFLAGS=''} to enable)
    \item Loop reordering with OpenMP on the CPU
    \item Loop reordering with OpenACC on the GPU
    \item OpenACC + CUBLAS
    \end{itemize}
  \end{itemize}
\end{frame}

\end{document}
