\documentclass[12pt,aspectratio=169]{beamer}

\usetheme{CSCS}

% define footer text
\newcommand{\footlinetext}{CSCS-USI Summer School 2017}

% Select the image for the title page
\newcommand{\picturetitle}{cscs_images/image3.pdf}

\lstdefinestyle{shstyle}{
  basicstyle=\scriptsize\ttfamily,
  keywordstyle=\color{blue},
  stringstyle=\color{magenta},
  commentstyle=\itshape\color{cscsred},
  language=bash,
}

\newcommand\shinline[2][]{\lstinline[style=shstyle,basicstyle=\ttfamily,#1]!#2!}

% Please use the predifined colors:
% cscsred, cscsgrey, cscsgreen, cscsblue, cscsbrown, cscspurple, cscsyellow,
% cscsblack, cscswhite

\author{Vasileios Karakasis, CSCS}
\title{Introduction to OpenACC}
\subtitle{Summer School 2017 -- Effective High Performance Computing}
\date{July 26, 2017}

\begin{document}

% TITLE SLIDE
\cscstitle

\begin{frame}{Goals of the course}
  \begin{itemize}
  \item Part I
    \begin{itemize}
    \item Overview of OpenACC
    \item Deeper understanding of the concepts through hands-on examples
    \end{itemize}
    \vspace\baselineskip
  \item  Part II
    \begin{itemize}
    \item Port the miniapp to GPU using OpenACC
    \item Walk away ready to start hacking your own code
    \end{itemize}
  \end{itemize}
\end{frame}

\begin{frame}{What is OpenACC?}
  \begin{itemize}
  \item Collection of compiler directives for specifying loops and regions to be
    offloaded from a host CPU to an attached accelerator device
  \item Host + Accelerator programming model
  \item High-level representation
  \item Current specification version: 2.5
    \begin{itemize}
    \item 2.6 is scheduled soon
    \end{itemize}
  \end{itemize}
\end{frame}

\begin{frame}{When to use OpenACC?}
  In any of the following cases:
  \vspace\baselineskip
  \begin{itemize}
  \item I program in Fortran
  \item I need portability across different accelerator vendors
  \item I don't care about the details, I want my science done
  \item I want to run on accelerators, but I still need a readable code
  \item I inhereted a large legacy monolithic codebase, which I don't dare to
    refactor completely, but I need results faster
  \end{itemize}
\end{frame}

\begin{frame}{OpenACC is not a silver bullet}
  \begin{itemize}
  \item User base is still relatively small but expanding
    \begin{itemize}
    \item You may run into compiler bugs or specification ambiguities
    \end{itemize}
  \item A high-level representation is not a panacea
    \begin{itemize}
    \item You need to adapt to the programming model
    \end{itemize}
  \item Be ware of avoiding a \lstinlineCpp{\#pragma}-clutter
    \begin{itemize}
    \item Rethink and refactor
    \end{itemize}
  \item Does not substitute hand-tuning
  \end{itemize}
\end{frame}

\begin{frame}[fragile]{Format of directives}
  \begin{itemize}
  \item C/C++
    \begin{itemize}
    \item \lstinlineCpp{\#pragma acc} \emph{directive-name [clause-list]
      new-line}
    \item Scope is the following block of code
    \end{itemize}
  \item Fortran
    \begin{itemize}
    \item \lstinlineFortran{\!\$acc} \emph{directive-name [clause-list]
      new-line}
    \item Scope is until \lstinlineFortran{\!\$acc end} \emph{directive-name}
    \end{itemize}
  \end{itemize}
\end{frame}

\begin{frame}{Programming model}
  \begin{itemize}
  \item Host-directed execution
  \item Compute intensive regions are offloaded to attached accelerator devices
  \item Host orchestrates the execution on the device
    \begin{itemize}
    \item Allocations on the device
    \item Data transfers
    \item Kernel launches
    \item Wait for events
    \item Etc\dots
    \end{itemize}
  \end{itemize}
\end{frame}

\begin{frame}{Execution model}
  \begin{itemize}
  \item The device executes \emph{parallel} or \emph{kernel regions}
  \item Parallel region
    \begin{itemize}
    \item Work-sharing loops
    \end{itemize}
  \item Kernel region
    \begin{itemize}
    \item Multiple loops to be executed as multiple kernels
    \end{itemize}
  \item Levels of parallelim
    \begin{enumerate}
    \item \emph{Gang} \onslide<2->{\emph{$\rightarrow$ CUDA block}}
    \item \emph{Worker} \onslide<2->{\emph{$\rightarrow$ CUDA warp or second dimension of a block}}
    \item \emph{Vector} \onslide<2->{\emph{$\rightarrow$ CUDA threads}}
    \end{enumerate}
    \begin{itemize}
    \item Parallelism levels are decided by the compiler but can be fine-tuned by
      the user
    \onslide<2->{
      {\color{cscsred} \item Mapping to CUDA blocks/warps/threads is implementation defined}
    }
    \end{itemize}
  \end{itemize}
\end{frame}

\begin{frame}{Execution model}{Modes of execution}
  \begin{itemize}
  \item Gang
    \begin{itemize}
    \item Gang-redundant (GR)
    \item Gang-partioned (GP)
    \end{itemize}
  \item Worker
    \begin{itemize}
    \item Worker-single (WS)
    \item Worker-partitioned (WP)
    \end{itemize}
  \item Vector
    \begin{itemize}
    \item Vector-single (VS)
    \item Vector-partitioned (VP)
    \end{itemize}
  \end{itemize}
\end{frame}

\begin{frame}[fragile]{Execution model}{The \lstinlineCpp{kernels} construct}
  \begin{Fortranlisting}{Multiple loops inside kernels construct}
!$acc kernels
    !GR mode
    do i = 1, N
        !compiler decides on the partitioning (GP/WP/VP modes)
        y(i) = y(i) + a*x(i)
    enddo
    do i = 1, N
        !compiler decides on the partitioning (GP/WP/VP modes)
        y(i) = b*y(i) + a*x(i)
    enddo
!$acc end kernels
  \end{Fortranlisting}
  \begin{itemize}
  \item Compiler will try to deduce parallelism
  \item Loops are launched as different GPU kernels
  \end{itemize}
\end{frame}

\begin{frame}[fragile]{Execution model}{The \lstinlineCpp{parallel} construct}
  \begin{Fortranlisting}{Parallel construct}
!$acc parallel
    do i = 1, N
        ! loop executed in GR mode
        y(i) = y(i) + a*x(i)
    enddo
    !$acc loop
    do i = 1, N
        !compiler decides on the partitioning (GP/WP/VP modes)
        y(i) = b*y(i) + a*x(i)
    enddo
!$acc end parallel
  \end{Fortranlisting}
  \begin{itemize}
  \item No automatic parallelism deduction $\rightarrow$ parallel loops must
    be specified explicitly
  \item Implicit gang barrier at the end of \lstinlineCpp{parallel}
  \end{itemize}
\end{frame}

\begin{frame}{Execution model}{Work-sharing loops}
  \begin{itemize}
  \item C/C++: \lstinlineCpp{\#pragma acc loop}
    \begin{itemize}
    \item Applies to the immediately following \lstinlineCpp{for} loop
    \end{itemize}
  \item Fortran: \lstinlineFortran{\!\$acc loop}
    \begin{itemize}
    \item Applies to the immediately following \lstinlineCpp{do} loop
    \end{itemize}
  \item Loop will be automatically striped and assigned to different threads
    \begin{itemize}
    \item Use the \lstinlineCpp{independent} clause to force striping
    \end{itemize}
  \item Convenience syntax combines
    \lstinlineCpp{parallel}/\lstinlineCpp{kernels} and \lstinlineCpp{loop}
    constructs
    \begin{itemize}
    \item \lstinlineCpp{\#pragma acc parallel loop}
    \item \lstinlineCpp{\#pragma acc kernels loop}
    \item \lstinlineFortran{\!\$acc parallel loop}
    \item \lstinlineFortran{\!\$acc kernels loop}
    \end{itemize}
  \end{itemize}
\end{frame}

\begin{frame}[fragile]{Execution model}{Work-sharing loops -- the \lstinlineCpp{collapse} clause}
  \begin{Fortranlisting}{Collapse loops}
!$acc loop collapse(2)
do i = 1,N
    do j = 1,N
        A(i,j) = coeff*B(i,j)
    enddo
enddo
  \end{Fortranlisting}
  \begin{itemize}
  \item OpenACC vs.\ OpenMP
    \begin{itemize}
    \item OpenACC: apply the \lstinlineCpp{loop} directive to the following $N$
      loops and possibly collapse their iteration spaces if independent
    \item OpenMP: Collapse the iteration spaces of the following $N$ loops
    \end{itemize}
  \end{itemize}
\end{frame}

\begin{frame}[fragile]{Execution model}{Controlling parallelism}
  \begin{itemize}
  \item Amount of parallelism at the \lstinlineCpp{kernels} and
    \lstinlineCpp{parallel} level
    \begin{itemize}
    \item \lstinlineCpp{num_gangs(...)}, \lstinlineCpp{num_workers(...)},
      \lstinlineCpp{vector_length(...)}
    \end{itemize}
  \item At the \lstinlineCpp{loop} level
    \begin{itemize}
    \item \lstinlineCpp{gang}, \lstinlineCpp{worker}, \lstinlineCpp{vector}
    \end{itemize}
  \end{itemize}

  \begin{Fortranlisting}{100 thread blocks with 128 threads each}
!$acc parallel num_gangs(100), vector_length(128)
    !$acc loop gang, vector
    do i = 1, n
        y(i) = y(i) + a*x(i)
    enddo
!$acc end parallel
  \end{Fortranlisting}
\end{frame}

\begin{frame}[fragile]{Execution model}{Variable scoping}
  \begin{itemize}
  \item Allowed in the \lstinlineCpp{parallel} directive only
  \item By default, if outside of a code block, variables are shared in global memory
  \item \lstinlineCpp{private}: A copy of the variable is placed in each \emph{gang} (CUDA block)
  \item \lstinlineCpp{firstprivate}: Same as \lstinlineCpp{private} but initialized from the host value
  \end{itemize}
  \pause
  Implicit scoping:
  \begin{itemize}
  \item (C/C++/Fortran) Loop variables are private to the \emph{thread} that executes the loop
  \item (C/C++ only) Scope of variables declared inside a parallel block depends on the current execution mode:
    \begin{itemize}
    \item \emph{Vector-partitioned} mode $\rightarrow$ private to the thread
    \item \emph{Worker-partitioned, Vector-single} mode $\rightarrow$ private to the worker
    \item \emph{Worker-single} mode $\rightarrow$ private to the gang
    \end{itemize}
  \end{itemize}
\end{frame}

\begin{frame}[fragile]{Execution model}{Reduction operations}
  \begin{itemize}
  \item \lstinlineCpp{\#pragma acc parallel reduction(<op>:<var>)}
    \begin{itemize}
    \item e.g., \lstinlineCpp{\#pragma acc parallel reduction(+:sum)}
    \end{itemize}
  \item \lstinlineCpp{\#pragma acc loop reduction(<op>:<var>)}
  \item \lstinlineCpp{var} must be scalar
  \item \lstinlineCpp{var} is copied and default initialized within each gang
  \item Intermediate results from each gang are combined and made available outside the parallel region
  \item Complex numbers are also supported
  \item Operators: \lstinlineCpp{+}, \lstinlineCpp{*}, \lstinlineCpp{max}, \lstinlineCpp{min}, \lstinlineCpp{&}, \lstinlineCpp{|}, \lstinlineCpp{\%}, \lstinlineCpp{&&}, \lstinlineCpp{||}
  \end{itemize}
\end{frame}

\begin{frame}[fragile]{Execution model}{Calling functions from parallel regions}
  \begin{itemize}
  \item \lstinlineCpp{\#pragma acc routine \{gang | worker | vector | seq\}}
    \begin{itemize}
    \item Just before the function declaration or definition
    \end{itemize}
  \item \lstinlineFortran{\!\$acc routine \{gang | worker | vector | seq\}}
    \begin{itemize}
    \item In the specification part of the subroutine
    \end{itemize}
  \item Parallelism level of the routine
    \begin{itemize}
    \item \lstinlineCpp{gang}: must be called from GR context
    \item \lstinlineCpp{worker}: must be called from WS context
    \item \lstinlineCpp{vector}: must be called from VS context
    \item \lstinlineCpp{seq}: must be called from sequential context
    \end{itemize}
  \end{itemize}
\end{frame}

\begin{frame}[fragile]{Memory model}{Where is my data?}
  \begin{itemize}
  \item The host and the device have separate address spaces
    \begin{itemize}
    \item Data management between the host and the device is the programmer's responsibility
    \item You must make sure that all the necessary data for a computation is available on the accelerator before entering the compute region
    \item You must make sure to transfer the processed data back to the host if needed
    \end{itemize}
    \pause
  \item But there can be some exceptions:
    \begin{itemize}
    \item The ``device'' might be the multicore $\rightarrow$ no need for data management
    \item Some compilers may infer automatically the necessary data transfers
    \item Nvidia Pascal GPUs provide efficient support for a unified memory view between the host and the accelerator
    \end{itemize}
  \end{itemize}
\end{frame}

\begin{frame}[fragile]{Memory model}{Directives accepting data clauses}
  Data clauses may appear in the following directives:
    \vspace\baselineskip
  \begin{itemize}
  \item Compute directives:
    \begin{itemize}
    \item\lstinlineCpp{#pragma acc kernels}
    \item\lstinlineCpp{#pragma acc parallel}
    \end{itemize}
    \vspace{.5\baselineskip}
  \item Data directives:
    \begin{itemize}
    \item \lstinlineCpp{#pragma acc data}
    \item \lstinlineCpp{#pragma acc enter data}
    \item \lstinlineCpp{#pragma acc exit data}
    \item \lstinlineCpp{#pragma acc declare}
    \item \lstinlineCpp{#pragma acc update}
    \end{itemize}
  \end{itemize}
\end{frame}

\begin{frame}[fragile]{Memory model}{Data clauses}
  \begin{itemize}
  \item \lstinlineCpp{create(a[0:n])}: Allocate array \lstinlineCpp{a} on device
  \item \lstinlineCpp{copyin(a[0:n])}: Copy array \lstinlineCpp{a} to device
  \item \lstinlineCpp{copyout(a[0:n])}: Copy array \lstinlineCpp{a} from device
  \item \lstinlineCpp{copy(a[0:n])}: Copy array \lstinlineCpp{a} to and from device
  \item \lstinlineCpp{present(a)}: Inform OpenACC runtime that array \lstinlineCpp{a} is on device
  \item \lstinlineCpp{delete(a)}: Deallocate array \lstinlineCpp{a} from device (\lstinlineCpp{exit data} only)
  \end{itemize}
  \vfill
  Not for the \shinline{acc update} directive
\end{frame}

\begin{frame}[fragile]{Memory model}{The \texttt{acc data} directive}
  \begin{itemize}
  \item Defines a scoped data region
    \begin{itemize}
    \item Data will be copied in at entry of the region and copied out at exit
    \item A \emph{structural reference count} is associated with each memory region that appears in the data clauses
    \end{itemize}
    \vfill
  \item C/C++: \lstinlineCpp{#pragma acc data} \emph{\texttt{[data clauses]}}
    \begin{itemize}
    \item The next block of code is a data region
    \end{itemize}
    \vfill
  \item Fortran: \lstinlineFortran{\!\$acc data} \emph{\texttt{[data clauses]}}
    \begin{itemize}
    \item Defines a data region until \lstinlineFortran{\!\$acc end data} is encountered
    \end{itemize}
  \end{itemize}
\end{frame}

\begin{frame}[fragile]{Memory model}{The \texttt{acc enter/exit data} directives}
  \begin{itemize}
  \item Defines an unscoped data region
    \begin{itemize}
    \item Data will be resident on the device until a corresponding \lstinlineCpp{exit data} directive is found
    \item Useful for managing data on the device across compilation units
    \item A \emph{dynamic reference count} is associated with each memory region that appears in the data clauses
    \end{itemize}
    \vfill
  \item C/C++:
    \begin{itemize}
    \item \lstinlineCpp{#pragma acc enter data} \emph{\texttt{[data clauses]}}
    \item\lstinlineCpp{#pragma acc exit data} \emph{\texttt{[data clauses]}}
    \end{itemize}
    \vfill
  \item Fortran:
    \begin{itemize}
    \item \lstinlineFortran{\!\$acc enter data} \emph{\texttt{[data clauses]}}
    \item \lstinlineFortran{\!\$acc exit data} \emph{\texttt{[data clauses]}}
    \end{itemize}
  \end{itemize}
\end{frame}

\begin{frame}[fragile]{Memory model}{The \texttt{acc declare} directive}
  \begin{itemize}
  \item Functions, subroutines and programs define \emph{implicit data regions}
  \item The \lstinlineCpp{acc declare} directive is used in variable declarations for making them available on the device during the lifetime of the implicit data region
  \item Useful for copying global variables to the device
    \vfill
  \item C/C++: \lstinlineCpp{#pragma acc declare} \emph{\texttt{[data clauses]}}
  \item Fortran: \lstinlineFortran{\!\$acc declare} \emph{\texttt{[data clauses]}}
  \end{itemize}
\end{frame}

\begin{frame}[fragile]{Memory model}{The \texttt{acc update} directive}
  \begin{itemize}
  \item May be used during the lifetime of device data for updating the copies on either host or the device
    \vfill
  \item \lstinlineCpp{#pragma acc update host(<var-list>)}
    \begin{itemize}
    \item Update host copy with corresponding data from the device
    \end{itemize}
    \vfill
  \item \lstinlineCpp{#pragma acc update device(<var-list>)}
    \begin{itemize}
    \item Update device copy with corresponding data from the host
    \end{itemize}
  \end{itemize}
\end{frame}

\begin{frame}[fragile]{Memory model}{Array ranges}
  Data clauses may accept as arguments
  \begin{itemize}
  \item Whole arrays
    \begin{itemize}
    \item C/C++: You \emph{must} specify bounds for dynamically allocated arrays
      \begin{itemize}
      \item \lstinlineCpp{\#pragma acc data copyin(a[0:n])}
      \item But \lstinlineCpp{\#pragma acc data present(a)} is acceptable: \lstinlineCpp{a}'s bounds can be inferred by the runtime
      \end{itemize}
    \item Fortran: array shape information is already embedded in the data type
      \begin{itemize}
      \item \lstinlineFortran{\!\$acc data copyin(a)}
      \end{itemize}
    \end{itemize}
  \item Array subranges
    \begin{itemize}
    \item \lstinlineCpp{\#pragma acc data copyin(a[2:n-2])}
    \end{itemize}
  %% \item Hint that a subarray should reside in the shared memory of the device
  %%   \begin{itemize}
  %%   \item \lstinlineCpp{\#pragma acc cache(<varlist>)}
  %%   \end{itemize}
  \end{itemize}
\end{frame}

\begin{frame}[fragile]{Advanced topics}{Synchronization primitives}
  \begin{itemize}
  \item Atomic operations
    \begin{itemize}
    \item \lstinlineCpp{\#pragma acc atomic} [atomic-clause]
    \item \lstinlineFortran{\!\$acc atomic} [atomic-clause]
    \item Atomic clauses: \lstinlineCpp{read}, \lstinlineCpp{write},
      \lstinlineCpp{update} and \lstinlineCpp{capture}
    \item Example of ``capturing'' a value:
      \begin{itemize}
      \item \lstinlineCpp{v = x++;}
      \end{itemize}
    \end{itemize}
  \item No global barriers $\rightarrow$ cannot be implemented due to hardware restrictions
  \item No equivalent \lstinlineCpp{__syncthreads()}
  \end{itemize}
\end{frame}

\begin{frame}[fragile]{Advanced topics}{Activity queues}
  \begin{itemize}
  \item Activity queues are the equivalent of CUDA event queues or streams
  \item Data copies and kernels are launched \emph{synchronously} inside the activity queues
  \item Additional clauses in compute or data directives control the activity queues:
    \begin{itemize}
    \item \lstinlineCpp{async(<qno>)}: push operations to activity queue \texttt{qno} and continue execution on the host
    \item \lstinlineCpp{wait(<qno>)}: wait for pending operations in activity queue \texttt{qno} to finish before launching next operation on the device
    \end{itemize}
  \item \lstinlineCpp{\#pragma acc wait(<qno>)}: Wait for all events in activity queue \texttt{qno} to finish before continuing execution on the host
  \end{itemize}
\end{frame}


\begin{frame}[fragile]{Advanced topics}{Activity queues example}
  \begin{Cpplisting}{Launch multiple kernels asynchronously on the GPU}
// Launch kernel on GPU and continue on CPU
#pragma acc parallel loop async(1) present(a)
for(i = 0; i < N; ++i) {
  a[i] = // ... compute on GPU
}
// Launch another kernel on GPU and continue on CPU
#pragma acc parallel loop async(2) present(b)
for(j = 0; j < N; ++j) {
  b[j] = // ... compute on GPU
}
// Wait for all kernels to finish
#pragma acc wait
  \end{Cpplisting}
  \begin{itemize}
  \item Especially useful for overlapping data transfers and execution
  \end{itemize}
\end{frame}

\begin{frame}[fragile]{Advanced topics}{Unified memory}
  \begin{itemize}
  \item Virtual address space shared between CPU and GPU
  \item The CUDA driver and the hardware take care of the page migration
  \item Introduced with the Kepler architecture and CUDA 6, but is significantly improved with Pascal
    \pause\vfill
  \item You could completely omit the data management in OpenACC !
  \item Currently, supported by the PGI compiler using the \shinline{-ta=tesla:unified} option
  \end{itemize}
\end{frame}

\begin{frame}[fragile]{Advanced topics}{Deep copy}
  \vspace{.5em}
  \begin{Cpplisting}{Deep copy example}
struct foo {
    int *array;
    size_t len;
};
foo a[10];
for (int i = 0; i < 10; ++i) {
    a.len = 100;
    a.array = new int[a.len];
}
#pragma acc enter data copyin(a[0:10])
  \end{Cpplisting}
  \begin{itemize}
  \item What will be copied over to the device? \onslide<2->{$\rightarrow$ just
    \lstinlineCpp{a} with dangling \lstinlineCpp{array} pointers :-(}
  \item<3-> What you would like to be copied? $\rightarrow$
    everything, you must wait for OpenACC 3.0
    \begin{itemize}
    \item<3->Cray compiler supports deep copy of derived types in Fortran only
    \item<3->PGI compiler introduced support for manual deep copy
    \end{itemize}
  \end{itemize}
\end{frame}


\begin{frame}[fragile]{Combining it all}
  \begin{Cpplisting}{Data movement/Activity queues/Parallel loops}
// prepare array a on host
#pragma acc enter data async(1) copyin(a[0:N])
// prepare array b on host
#pragma acc enter data async(2) copyin(b[0:N])
#pragma acc parallel loop async(1) present(a[0:N])
for (i = 0; i < N: ++i)
    foo(a[i])

#pragma acc exit data copyout(a[0:N]) async(1)
#pragma acc parallel loop async(2) present(b[0:N])
for (i = 0; i < N; ++i)
    bar(b[i])
#pragma acc exit data copyout(b[0:N]) async(2)
// some more stuff on the host and then wait for all streams to finish
#pragma acc wait

  \end{Cpplisting}
\end{frame}

\begin{frame}[fragile]{Profiling}
  \begin{itemize}
  \item NVIDIA tools (\shinline{nvprof}, \shinline{nvpp})
    \begin{itemize}
    \item \shinline{\$ nvprof <openacc-executable>}
    \end{itemize}
    \vspace\baselineskip
  \item CrayPAT
    \begin{itemize}
    \item \shinline{\$ module load daint-gpu}
    \item \shinline{\$ module load perftools-cscs/645openacc}
    \item Recompile and run
    \item Report in \lstinlineCpp{.rpt} file
    \end{itemize}
  \end{itemize}
\end{frame}

\begin{frame}[fragile]{Hands-on exercises}{General information}
  \begin{itemize}
  \item \shinline{grep TODO *.\{cpp,f90,f03\}}
  \item Both Cray/PGI compilers are supported, unless otherwise stated
  \item \shinline{source <ssprefix>/scripts/setup.sh} $\rightarrow$ will make available PGI 17.4
  \item \shinline{module load craype-accel-nvidia60}
  \item \shinline{module switch pgi/16.9.0 pgi/17.4} (after \texttt{PrgEnv-pgi} is loaded)
  \item \shinline{make} or \shinline{make VERBOSE=1} to get compiler information about offloaded regions
  \end{itemize}
\end{frame}

\begin{frame}{Hands-on}{The basics}
  \begin{itemize}
  \item Vector scale:
    \begin{itemize}
    \item \shinline{exercises/openacc/shared/axpy_openacc.\{cpp,f90\}}
    \item Run as: \\
      \shinline{srun --reserv=summer -Cgpu ./axpy.openacc [ARRAY\_SIZE]}
    \item \shinline{ARRAY\_SIZE} is power of 2, default is 16
    \end{itemize}
    \vfill
  \item Dot product:
    \begin{itemize}
    \item \shinline{exercises/openacc/shared/dot_openacc.\{cpp,f90\}}
    \end{itemize}
  \end{itemize}
\end{frame}

\begin{frame}{Data management}
  \begin{itemize}
  \item Moving data to and from the device is slow ($\approx$7--8\,GB/s per direction)
  \item Avoid unnecessary data movement
    \begin{itemize}
    \item Move needed data to GPU early enough and keep it there as long as possible
    \item Update host copies using \lstinlineCpp{\#pragma acc update} directive if needed
    \end{itemize}
  \end{itemize}
\end{frame}

\begin{frame}[fragile]{Hands-on}{Blur kernel}
  \begin{Cpplisting}{Naive implementation}
for (auto istep = 0; istep < nsteps; ++istep) {
    int i;

    #pragma acc parallel loop copyin(in[0:n]) copyout(buffer[0:n])
    for(i=1; i<n-1; ++i) {
        buffer[i] = blur(i, in);
    }
    #pragma acc parallel loop copyin(buffer[0:n]) copy(out[0:n])
    for(i=2; i<n-2; ++i) {
        out[i] = blur(i, buffer);
    }

    std::swap(in, out);
}
  \end{Cpplisting}
\end{frame}

\begin{frame}[fragile]{Interoperability with MPI and CUDA}
  \begin{enumerate}
  \item Call an optimised library function that expects data on the device, e.g., cuBLAS
  \item Let optimised MPI implementations do RDMA between remote devices' memory
  \item Manual data management with CUDA, but parallelisation with OpenACC
    \begin{itemize}
    \item The safest way to manipulate pointers on the device
    \end{itemize}
  \end{enumerate}
  \onslide<2->{
    \begin{red2block}{Scenarios (1) and (2)}
      \lstinlineCpp{\#pragma acc host\_data use\_device(<varlist>)}
    \end{red2block}
  }
  \onslide<3->{
    \begin{red2block}{Scenario (3)}
      Use the \lstinlineCpp{deviceptr(<ptrlist>)} clause with
      \lstinlineCpp{parallel}, \lstinlineCpp{kernels} and \lstinlineCpp{data}
    \end{red2block}
  }
\end{frame}

\begin{frame}[fragile]{Hands-on}{2D diffusion example}
  Source code:
  \begin{itemize}
  \item \shinline{diffusion2d\_omp.\{cpp,f90\}}: our baseline code
    \begin{itemize}
    \item Single node OpenMP version for the CPU
    \end{itemize}
  \item \shinline{diffusion2d\_openacc.\{cpp,f90\}}
    \begin{itemize}
    \item Single node OpenACC version
    \end{itemize}
  \item \shinline{diffusion2d\_openacc\_mpi.\{cpp,f90\}}
    \begin{itemize}
    \item MPI+OpenACC version
    \item If \shinline{OPENACC\_DATA} is undefined, data management is performed by CUDA
    \end{itemize}
  \end{itemize}
\end{frame}


\begin{frame}[fragile]{Outlook}
  OpenACC 2.6 is due end of the year
  \begin{itemize}
  \item Manual deep copy
  \item Standardize behavior of Fortran optional arguments
  \item Fortran bindings for all API routines
  \item \lstinlineCpp{acc serial} directive
  \item Device query routines
  \item Improvements in error handling
  \end{itemize}
  OpenACC 3.0
  \begin{itemize}
  \item Not scheduled yet
  \item The big feature should be the true deep copy
  \end{itemize}
\end{frame}


\begin{frame}{OpenACC vs.\ OpenMP}
  \begin{itemize}
  \item OpenMP 4.0 introduced directives for offloading computation to accelerators
  \item Similar concepts to OpenACC but OpenMP is a more prescriptive standard
  \item There is no OpenMP-OpenACC merger envisioned right now
  \end{itemize}

  \vfill
  OpenACC and compiler support
  \begin{itemize}
  \item PGI
    \begin{itemize}
    \item Latest spec support; drives the OpenACC development
    \end{itemize}
  \item Cray
    \begin{itemize}
    \item Support up to OpenACC 2.0; no new features or later spec support
    \item Bug fixes and support for the current implementation only
    \end{itemize}
  \item GCC
    \begin{itemize}
    \item Support of OpenACC 2.0a from GCC 5.1
    \item Support of OpenACC 2.5 perhaps in GCC 8.0
    \end{itemize}
  \end{itemize}
\end{frame}

\begin{frame}{More information and events}
  \begin{itemize}
  \item \url{http://www.openacc.org}
    \begin{itemize}
    \item Specification and related documents
    \item Tutorials
    \item Events
    \end{itemize}
    \vfill
  \item OpenACC Hackathons
    \begin{itemize}
    \item One week of intensive development for porting your code to the GPUs
    \item 3 developers + 2 mentors per team
    \item 3$\times$ in USA + 2$\times$ in Europe in 2017
    \item Find the one that fits you and apply!
    \end{itemize}
  \end{itemize}
\end{frame}

\part{Porting the miniapp to GPUs using OpenACC}

\begin{frame}[fragile]{General info}
  \begin{itemize}
  \item Fortran 90 version
    \begin{itemize}
    \item \lstinlineCpp{miniapp/openacc/fortran/}
    \end{itemize}
  \item C++11 version
    \begin{itemize}
    \item \lstinlineCpp{miniapp/openacc/cxx/}
    \item Compile with PGI 17.4
      \begin{itemize}
        \item\shinline{source <ss-prefix>/scripts/setup.sh}
        \item\shinline{module switch pgi/16.9.0 pgi/17.4}
      \end{itemize}
    \end{itemize}
  \item Interesting files
    \begin{itemize}
    \item\shinline{main.\{cpp,f90\}}: the solver
    \item\shinline{data.\{h,f90\}}: domain types
    \item\shinline{linalg.\{cpp,f90\}}: linear algebra kernels
    \item\shinline{operators.\{cpp,f90\}}: the diffusion kernel
    \end{itemize}
  \end{itemize}
\end{frame}

\begin{frame}{Notes for the C++ version}
  \begin{itemize}
  \item There are two C++-isms that complicate things:
    \begin{enumerate}
    \item Domain data is encapsulated inside the \lstinlineCpp{Field} class
      \begin{itemize}
      \item Allocated and initialised inside the constructor
      \item Deallocated inside the destructor
      \end{itemize}
    \item Operators for accessing the domain data
    \end{enumerate}
    \vfill
  \item OpenACC provides the \lstinlineCpp{enter data} and \lstinlineCpp{exit
    data} directives for unscoped data management
  \item Operators are just another kind of functions
    \begin{itemize}
    \item \lstinlineCpp{acc routine} directive is just for that
    \end{itemize}
  \item Remember to copy the object itself (\lstinlineCpp{this} pointer)
  \end{itemize}
\end{frame}

\end{document}
