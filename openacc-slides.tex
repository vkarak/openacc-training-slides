\documentclass[12pt,aspectratio=169]{beamer}

\usetheme{CSCS}

% define footer text
\newcommand{\footlinetext}{CSCS-USI Summer School 2016}

% Select the image for the title page
\newcommand{\picturetitle}{cscs_images/image3.pdf}


% Please use the predifined colors:
% cscsred, cscsgrey, cscsgreen, cscsblue, cscsbrown, cscspurple, cscsyellow,
% cscsblack, cscswhite

\author{Vasileios Karakasis, CSCS}
\title{Introduction to OpenACC}
\subtitle{Summer School 2016 -- Effective High Performance Computing}
\date{July 27, 2016}

\begin{document}

% TITLE SLIDE
\cscstitle

% TABLE OF CONTENT SLIDE
% All options for table of contents:
% currentsection, currentsubsection, firstsection=xx, hideallsubsections, hideothersubsections, part=xx, pausesections, pausesubsections, sections=xx, sections={xx-yy}, sections={xx,yy}
%\cscstableofcontents[hideallsubsections]{Title}
\cscstableofcontents{Goals of this course}

% Pseudo-structure in order just to get into the TOC
\section{Part I}
\subsection{Quick overview of OpenACC}
\subsection{Deeper understanding of the concepts through hands-on examples}

\section{Part II}
\subsection{Port the miniapp to GPU using OpenACC}
\subsection{Walk away ready (or willing) to start hacking your own code}

\begin{frame}{What is OpenACC?}
  \begin{itemize}
  \item Collection of compiler directives for specifying loops and regions to be
    offloaded from a host CPU to an attached accelerator device
  \item Host + Accelerator programming model
  \item High-level representation
  \end{itemize}
\end{frame}

\begin{frame}{When to use OpenACC?}
  \begin{itemize}
  \item I program in Fortran
  \item I need portability across different accelerator vendors
  \item I don't care about the details, I want my science done
  \item I want to run on accelerators, but I still need a readable code
  \item I inhereted a large legacy monolithic codebase, which I don't dare to
    refactor completely, but I need results faster
  \end{itemize}
\end{frame}

\begin{frame}{OpenACC is not a silver bullet}
  \begin{itemize}
  \item User base is still relatively small but expanding
    \begin{itemize}
    \item You may run into compiler bugs or specification ambiguities
    \end{itemize}
  \item A high-level representation is not a panacea
    \begin{itemize}
    \item You need to adapt to the programming model
    \end{itemize}
  \item Be ware of avoiding a \lstinlineCpp{\#pragma}-clutter
    \begin{itemize}
    \item Rethink and refactor
    \end{itemize}
  \item Does not substitute hand-tuning
  \end{itemize}
\end{frame}

% Programming model (data and compute regions)
% Form of the directives + reduction
% First hands-on example axpy + reduction
% Separate data regions and compute regions, Reduction
% Hands-on blur example (naive + no unnecessary copies)
% Interoperability with CUDA
% Hands-on diffusion
% OpenACC outlook and comparison with OpenMP 4 and 4.5

% Part II -- Porting the miniapp
% Talk about the structure of the application, especially the Field class
% Revisit the enter/exit data directives
% Describe the porting approach step-by-step
% 1. Field, 2. linalg, 3. operators, 4. rest

% THANK YOU SLIDE
\cscsthankyou{Thank you for your attention.}

\end{document}
